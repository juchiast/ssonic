\chapter{Sonic Polynomial Interactive Oracle Proof}
\label{chap-sonic}

\section{Sonic Constraints System}

This section describes the constraints system (a system of equations) used by Sonic \cite{maller2019sonic}, which we called \textit{Sonic Constraints System}. The constraints system operates on a finite field $\mathrm{GF}(p)$ for a large prime $p$, which we denote $\F$ omitting the order $p$. We define this constraints system as an NP relation.

\begin{definition}[Sonic Constraints System]
Given a finite field with prime order $\F$, two integers $n > 0$ and $Q \ge 0$, the Sonic Constraints System is defined by the relation $\RSonic$,
\[
\RSonic(\F, n, Q) = \left\{
\langle
    \{ \Vec{u}_q, \Vec{v}_q, \Vec{w}_q, k_q \}_{q=1}^{Q},
    (\Vec{a}, \Vec{b}, \Vec{c})
\rangle
:
\begin{array}{l}
    \forall i \in [n], \forall q \in [Q] \\
    \Vec{a}, \Vec{b}, \Vec{c} \in \F^n \\
    \Vec{u}_q, \Vec{v}_q, \Vec{w}_q \in \F^n \\
    a_i b_i = c_i \\
    \langle \Vec{a}, \Vec{u}_q \rangle
      + \langle \Vec{b}, \Vec{v}_q \rangle
      + \langle \Vec{c}, \Vec{w}_q \rangle
      = k_q
\end{array}
\right\}
\]
\end{definition}

For simplicity, sometime we omit $(\F, n, Q)$ and only write $\RSonic$. The witness of this relation is three vectors $\Vec{a}, \Vec{b}, \Vec{c} \in \F^n$, the instance is the set of $Q$ constraints $\{ \Vec{u}_q, \Vec{v}_q, \Vec{w}_q, k_q \}_{q=1}^{Q}$. We will later show how to convert an Arithmetic Circuit Satisfiability problem into this form of constraints, and also that we can convert from a Boolean Circuit to an Arithmetic Circuit  over a finite field $\mathrm{GF}(p)$, therefore, all NP statements can be transformed into the Sonic Constraints System.

It is worth knowing that, for a circuit satisfiability problem $\mathrm{C}(x, w) = 0$, the structure of $\mathrm{C}$ is captured by $\{ \Vec{u}_q, \Vec{v}_q, \Vec{w}_q \}$; $\{ k_q \}$ captures $x$ and constant values used in $\mathrm{C}$; $(\Vec{a}, \Vec{b}, \Vec{c})$ contains all wire's values of $\mathrm{C}$, and thus contains $w$.

We remarks that after conversion from Arithmetic Circuit Satisfiability, most scalar values in vectors $\Vec{u}_q, \Vec{v}_q, \Vec{w}_q$ are zero. In fact, each equation $\langle \Vec{a}, \Vec{u}_q \rangle + \langle \Vec{b}, \Vec{v}_q \rangle + \langle \Vec{c}, \Vec{w}_q \rangle = k_q$ can be written as $a_i u_i + b_j  v_j + c_k w_k = k_q$ for some $i, j, k \in [n]$ and $u_i, v_j, w_k \in \{-1, 0, 1\}$. Thus, we are actually using a very small subset of $\RSonic$, it might be possible to design a proof system for this subset and achieve better efficiency.
Nevertheless, the next section describe how to convert $\RSonic$ into probabilistic testing of a univariate polynomial.

\section{Sonic Constraint System to Polynomial Testing}

\begin{lemma}[Schwartz-Zippel lemma]
Let $\F$ be a finite field, let $f(X) \in \F[X]$ be a degree-$d$ polynomial on $\F$. Let $x$ is sampled uniformly random from $\F$, if $f(X) \ne 0$ then
$$\pr{f(x) = 0} \le \frac{d}{|\F|}$$
\end{lemma}

We want to transform $x = \{ \Vec{u}_q, \Vec{v}_q, \Vec{w}_q, k_q \}_{q=1}^{Q}$ and $w = (\Vec{a}, \Vec{b}, \Vec{c})$ into $f(Y)$ such that for all $x, w$, $(x, w) \in \RSonic \iff f(Y) = 0$ (we recall that writing $f(Y) = 0$ means $f(Y)$ is the zero polynomial).

Following the methodology in \cite{maller2019sonic}, we embed each equation into a coefficient of $f(Y)$:
\begin{itemize}
    \item For all $i \in [n]$, $a_i b_i = c_i$ becomes $(a_i b_i - c_i) Y^i = 0$ and $(a_i b_i - c_i) Y^{-i} = 0$, the second equation is redundant, it was added to support the transformation in the next section.
    \item For all $q \in [Q]$, $\langle \Vec{a}, \Vec{u}_q \rangle
      + \langle \Vec{b}, \Vec{v}_q \rangle
      + \langle \Vec{c}, \Vec{w}_q \rangle
      = k_q$ becomes $$(\langle \Vec{a}, \Vec{u}_q \rangle
      + \langle \Vec{b}, \Vec{v}_q \rangle
      + \langle \Vec{c}, \Vec{w}_q \rangle
      - k_q) Y^{n + q} = 0$$
\end{itemize}

Summing up the left-hand sides of the above equations, we have
\begin{align*}
f(Y) = & \sum_{i=1}^{n}(a_i b_i - c_i) Y^{-i} \\
& + \sum_{i=1}^{n}(a_i b_i - c_i) Y^i \\
& + \sum_{q=1}^{Q} (\langle \Vec{a}, \Vec{u}_q \rangle
      + \langle \Vec{b}, \Vec{v}_q \rangle
      + \langle \Vec{c}, \Vec{w}_q \rangle
      - k_q) Y^{n + q}
\end{align*}

With the above definition of $f(Y)$, we have $\forall x, y, (x, w) \in \RSonic \iff f(Y) = 0$, assuming a large enough field $\F$.

Having $f(Y)$ defined, we can use probabilistic testing of polynomials via the Schwartz-Zippel lemma. The Verifier chooses a random point $z \in \F$, queries for $f(z)$, and accepts if $f(z) = 0$, this protocol has soundness error $\mathrm{deg}(f(Y)) / |\F|$ according to the Schwartz-Zippel lemma. The main result of this section is summarized in claim \ref{sonic-univariate-test}.

\begin{claim}
\label{sonic-univariate-test}
Given a relation $\RSonic(\F, n, Q)$ for a large field $\F$. \\
Let $x = \{ \Vec{u}_q, \Vec{v}_q, \Vec{w}_q, k_q \}_{q=1}^{Q}$, $w = (\Vec{a}, \Vec{b}, \Vec{c})$. Let $f(Y)$ be defined by
\begin{align*}
f(Y) = & \sum_{i=1}^{n}(a_i b_i - c_i) Y^{-i} \\
& + \sum_{i=1}^{n}(a_i b_i - c_i) Y^i \\
& + \sum_{q=1}^{Q} (\langle \Vec{a}, \Vec{u}_q \rangle
      + \langle \Vec{b}, \Vec{v}_q \rangle
      + \langle \Vec{c}, \Vec{w}_q \rangle
      - k_q) Y^{n + q}
\end{align*}
We have that $\forall (x, w) \in \RSonic, f(Y) = 0$. \\
Let $z$ be chosen uniformly random from $\F$, then by the Schwartz-Zippel lemma we have
$$\pr{f(z) = 0 \mid (x, w) \not\in \RSonic } \le \frac{2n+Q}{|\F|}$$
\end{claim}

However, we cannot use $f(Y)$ directly in a Polynomial IOP. Although the Verifier can query the proof oracle $f(Y)$ sent by the Prover, he cannot check that $f(Y)$ is well-formed, in another word, he cannot form a relationship between the proof oracle $f(Y)$ and the instance $x = \{ \Vec{u}_q, \Vec{v}_q, \Vec{w}_q, k_q \}_{q=1}^{Q}$ being proved. To resolve this issue, we could try to split $f(Y)$ into two polynomials $f_1(Y)$ and $f_2(Y)$, such that $f_1(Y)$ contains the witness and $f_2(Y)$ contains the instance of the relation. For example, suppose $f(Y)$ can be factored as $f(Y) = f_1(Y) f_2(Y)$, then the Verifier can query for $f_1(z)$ and compute $f_2(z)$, then the check $f(z) = f_1(z) f_2(z)$ fail with overwhelming probability if $f(Y) \ne f_1(Y) f_2(Y)$. In fact, \cite{maller2019sonic} defines $R(X, Y)$, $S(X, Y)$ and $k(y)$ such that:
\begin{itemize}
    \item $R(X, Y)$ contains $(\Vec{a}, \Vec{b}, \Vec{c})$.
    \item $S(X, Y)$ contains $\{ \Vec{u}_q, \Vec{v}_q, \Vec{w}_q \}$.
    \item $k(Y)$ contains $\{ k_q \}$.
    \item The coefficient of $X^0$ of $R(X, 1) [R(X, Y) + S(X, Y)] - k(Y)$ is $f(Y)$.
\end{itemize}

The next section will explain how the Sonic paper \cite{maller2019sonic} transforms from $f(X)$ to $R(X, Y)$, $S(X, Y)$, and $k(Y)$.

\section{Splitting Witness and Instance from Polynomial Testing}

We reconstruct $t(X, Y)$ as in \cite{maller2019sonic}, the result of this section is summarized in claim \ref{sonic-rs}.

\begin{claim}
\label{sonic-rs}
Let $f(Y)$ be defined as in claim \ref{sonic-univariate-test}.
We define $R(X, Y)$, $S_1(X, Y)$, $S_2(X, Y)$, and $k(Y)$ as follow:
\begin{align*}
R(X, Y) &= \sum_{i=1}^n \left(a_i X^{-i} Y^{-i} + b_i X^{i} Y^{i} + c_i X^{-i-n} Y^{-i-n}\right) \\
k(Y) &= \sum_{q=1}^{Q} k_q Y^{n + q} \\
S_1(X, Y) &= \sum_{i=1}^{n} (Y^{i} + Y^{-i}) X^{n+i} \\
S_2(X, Y) &= \sum_{i=1}^{n} \left( u_i(Y)X^i + v_i(Y)X^{-i} + w_i(Y)X^{n+i}\right) \\
S(X, Y) &= S_1(X, Y) + S_2(X, Y)
\end{align*}
where, for $i \in [n]$,
\begin{align*}
u_i(Y) & = \sum_{q=1}^Q u_{q, i} Y^{q+n} \\
v_i(Y) & = \sum_{q=1}^Q v_{q, i} Y^{q+n} \\
w_i(Y) & = \sum_{q=1}^Q w_{q, i} Y^{q+n}    
\end{align*}

Finally, let $t(X, Y) = R(X, 1) \times \left[ R(X, Y) + S(X, Y) \right] - k(Y)$, we have that $f(Y)$ is the coefficient $X^0$ of $t(X, Y)$.
\end{claim}

Maller et al. \cite{maller2019sonic} uses a technique from Bootle et al. \cite{bootle2016efficient} to embed $f(Y)$ into the coefficient of $X^0$ of $t(X, Y)$, we summarize their technique here:
\begin{itemize}
    \item Let $f_a(X) = \sum_{i} a_i X^{i}$, $f_b(X) = \sum_{i} b_i X^{i}$ be polynomials that can contain negative exponents, the coefficient of $X^0$ of $f_a(X) f_b(X)$ is $\sum_{i} a_i b_{-i}$.
    \item Thus, one could compute the inner product of $\Vec{a}$ and $\Vec{b}$ by placing them in opposite sides of two polynomials and then multiply the two polynomials.
\end{itemize}

We need to re-write $f(Y)$ in claim \ref{sonic-univariate-test} as a sum of inner products. We have:
\begin{align*}
f(Y) = &
 \sum_{i=1}^{n} a_i \times b_i Y^i + \sum_{i=1}^{n} b_i \times a_i Y^{-i} \\
& + \sum_{i=1}^{n}c_i (-Y^i -Y^{-i}) \\
& + \sum_{q=1}^{Q} (\langle \Vec{a}, \Vec{u}_q \rangle
      + \langle \Vec{b}, \Vec{v}_q \rangle
      + \langle \Vec{c}, \Vec{w}_q \rangle
      - k_q) Y^{n + q} \\
= & f_1(Y) + f_2(Y) + f_3(Y) - k(Y)
\end{align*}

where
\begin{align*}
f_1(Y) & = \sum_{i=1}^{n} a_i \times b_i Y^i + \sum_{i=1}^{n} b_i \times a_i Y^{-i} \\
f_2(Y) & = \sum_{i=1}^{n}c_i (-Y^i -Y^{-i}) \\
f_3(Y) & = \sum_{q=1}^{Q} (\langle \Vec{a}, \Vec{u}_q \rangle
      + \langle \Vec{b}, \Vec{v}_q \rangle
      + \langle \Vec{c}, \Vec{w}_q \rangle) Y^{n + q} \\
k(Y) & = \sum_{q=1}^{Q} k_q Y^{n + q}
\end{align*}

We examine $f_1(Y)$, if we define $R(X, Y)$ as $R(X, Y) = \sum_{i=1}^n a_i Y^{-i} X^{-i} + \sum_{i=1}^n b_i Y^i X^{i}$, then the coefficient of $X^0$ of $R(X, 1) \times R(X, Y)$ is $f_1(Y)$.

Adding $f_2(Y)$ can be done by allocating $c_1,\dots,c_n$ in $X^{-n-1},\dots,X^{-n-n}$ of $R(X, Y)$, and introduce $S_1(X, Y) = \sum_{i=1}^{n} (Y^{i} + Y^{-i}) X^{n+i}$. However, we allocate $c_i Y^{-n-i}$ instead, then $R(X, Y) = \sum_{i=1}^n (a_i X^{-i} Y^{-i} + b_i X^{i} Y^{i} + c_i X^{-n-i} Y^{-n-i})$ has a nice property: we can evaluate a bivariate polynomial $R(X, Y)$ by evaluating a univariate polynomial $R(X, 1)$.

At this point, we have that $f_1(Y) + f_2(Y)$ is the coefficient of $X^0$ of $R(X, 1) R(X, Y) + R(X, 1) S_1(X, Y)$. We now examine $f_3(Y)$.
\begin{align*}
f_3(Y) & = \sum_{q=1}^{Q} (\langle \Vec{a}, \Vec{u}_q \rangle
      + \langle \Vec{b}, \Vec{v}_q \rangle
      + \langle \Vec{c}, \Vec{w}_q \rangle) Y^{n + q}
\end{align*}

Taking a look the inner product $\langle \Vec{a}, \Vec{u}_q \rangle$, we have
\begin{align*}
\sum_{q=1}^{Q} \langle \Vec{a}, \Vec{u}_q \rangle Y^{n + q}
& = \sum_{q=1}^{Q} \sum_{i=1}^{n} a_i \times u_{q, i} Y^{n + q} \\
& = \sum_{i=1}^{n} \left( a_i \times \sum_{q=1}^Q u_{q, i} Y^{q+n}\right) \\
& = \langle \Vec{a}, \Vec{u}(Y) \rangle
\end{align*}

where $\Vec{u}(Y)$ is a vector of $n$ polynomial $u_i(Y) = \sum_{q=1}^Q u_{q, i} Y^{q+n}$ for $i \in [n]$. Similarly, we define $\Vec{v}(Y)$ and $\Vec{w}(Y)$, for all $i \in [n]$
\begin{align*}
v_i(Y) & = \sum_{q=1}^Q v_{q, i} Y^{q+n} \\
w_i(Y) & = \sum_{q=1}^Q w_{q, i} Y^{q+n}    
\end{align*}

Recalling that $R(X, Y) = \sum_{i=1}^n \left(a_i X^{-i} Y^{-i} + b_i X^{i} Y^{i} + c_i X^{-n-i} Y^{-n-i}\right)$, we allocate
\begin{itemize}
    \item $u_1(Y),\dots,u_n(Y)$ in coefficients of $X^1,\dots,X^n$.
    \item $v_1(Y),\dots,v_n(Y)$ in coefficients of $X^{-1},\dots,X^{-n}$.
    \item $w_1(Y),\dots,w_n(Y)$ in coefficients of $X^{n+i},\dots,X^{n+i}$.
\end{itemize}

Finally, we have $S_2(X, Y)$ such that $f_3(Y)$ is the coefficient of $X^0$ of $R(X, 1) S_2(X, Y)$:
$$S_2(X, Y) = \sum_{i=1}^{n} \left( u_i(Y)X^i + v_i(Y)X^{-i} + w_i(Y)X^{n+i} \right)$$

We arrive at the result of Sonic \cite{maller2019sonic}, let
$$t(X, Y) = R(X, 1) \times \left[ R(X, Y) + S_1(X, Y) + S_2(X, Y) \right] - k(Y)$$
then $f(Y)$ if the coefficient of $X^0$ of $t(X, Y)$. Therefore, we reduced the problem probabilistic testing of $f(Y)$ to a coefficient query of $t(X, Y)$.

\section{Sonic Polynomial IOP}

With $R(X, Y)$ as witness and $S_1(X, Y)$, $S_2(X, Y)$, and $k(y)$ as instance, we need to prove that for a random challenge $y \in \F$, $t(X, y)$ has no constant term.

The idea of the protocol is summarized bellow:
\begin{enumerate}
    \item $\mathcal{P}$ sends proof oracle $R(X, 1)$.
    \item $\mathcal{V}$ samples a uniform random $y \in \F$.
    \item $\mathcal{P}$ sends proof oracle $t(X, y)$.
    \item $\mathcal{V}$ samples uniform random $z \in \F$, and queries for $R(z, 1)$, $R(zy, 1)$, $t(z, y)$. $\mathcal{V}$ checks that $t(z, y) = r(z,1) \times \left[ r(zy, 1) + S_1(z, y) + S_2(z, y)\right] - k(y)$
    \item $\mathcal{P}$ and $\mathcal{V}$ engages in a coefficient query protocol to prove that coefficient of $X^0$ of $t(X, y)$ is 0.
\end{enumerate}

This abstract oracle protocol will be compiled into an actual protocol using the DARK Commitment Scheme. However, we need some adjustments to be compatible with DARK \cite{bunz2020transparent}. DARK does not consider polynomials with negative exponents, therefore we want the proof oracles $R(X, 1)$ and $t(X, y)$ to have non-negative exponents.

Observing that the smallest exponent of $X$ of $R(X, Y)$ is $X^{-2n}$, we shift the degree of $X$ by $2n$:
\begin{align*}
R'(X, Y)
&= X^{2n} R(X, Y) \\
&= \sum_{i=1}^n \left( a_i Y^{-i} X^{-i + 2n} + b_i Y^{i} X^{i + 2n} + c_i Y^{-i-n} X^{-i+n}  \right)
\end{align*}

In the first step, $\mathcal{P}$ sends proof oracle $R'(X, 1)$ (instead of $R(X, 1)$).

We analyze $t(X, Y)$, recall that
$$t(X, Y) = R(X, 1) \times \left[ R(X, Y) + S(X, Y) \right] - k(Y)$$
Smallest exponent in $R(X, Y)$ is $X^{-2n}$ and $S(X, Y)$'s smallest is $X^{-n}$, therefore, smallest exponent of $X$ of $t(X, Y)$ is $X^{-4n}$, we shift $t(X, Y)$ by $X^{4n}$.
\begin{align*}
t'(X, Y)
&= X^{4n} t(X, Y) \\
&= X^{2n} R(X, 1) \left[ X^{2n} R(X, Y) + X^{2n} S(X, Y) \right] - X^{4n} k(Y) \\
&= R'(X, 1) \left[ R'(X, Y) + X^{2n} S(X, Y) \right] - X^{4n} k(Y)
\end{align*}

$\mathcal{P}$ will send $t'(X, y)$ instead of $t(X, y)$, $\mathcal{P}$ and $\mathcal{V}$ use coefficient query to prove that coefficient of $X^{4n}$ is 0. We conclude this section with the updated Sonic Polynomial IOP for $\RSonic$ in  protocol \ref{sonic-iop}.

\begin{protocol}[Sonic Polynomial IOP]
\label{sonic-iop}
\quad

\textup{Witness: $\Vec{a}, \Vec{b}, \Vec{c}$ \\
Instance: $\{ \Vec{u}_q, \Vec{v}_q, \Vec{w}_q, k_q \}_{q=1}^{Q}$ \\
Before entering the protocol, $\mathcal{P}$ computes $R'(X, Y)$ and $t'(X, Y)$, $\mathcal{P}$ and $\mathcal{V}$ compute $S(X, Y)$ and $k(Y)$.
\begin{enumerate}
    \item $\mathcal{P}$ sends proof oracle $R'(X, 1)$.
    \item $\mathcal{V}$ samples a uniform random $y \in \F$.
    \item $\mathcal{P}$ splits $t'(X, Y)$ at the term $X^{4n}$ into $t'_{L}(X, y)$ and $t'_{H}(X, y)$ such that $t'(X, y) = t'_L(x, y) + \alpha X^{4n} + X^{4n+1} t'_H(X, y)$. $\mathcal{P}$ sends proof oracle $t'(X, y)$, $t'_L(X, y)$, and $t'_H(X, y)$.
    \item $\mathcal{V}$ samples uniform random $z \in \F$, and queries for $R'(z, 1)$, $R'(zy, 1)$, $t'(z, y)$, $t'_L(z, y)$, $t'_H(z, y)$. $\mathcal{V}$ checks that $t(z, y) = r(z,1) \times \left[ r(zy, 1) + z^{2n} S(z, y)\right] - z^{4n}k(y)$ and $t'(z, y) = t'_L(z, y) + z^{4n+1} t'_H(z, y)$.
    \item $\mathcal{V}$ outputs 1 if all checks pass, otherwise outputs 0.
\end{enumerate}
}
\end{protocol}