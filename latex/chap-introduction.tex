\chapter{Introduction}
\label{chap-introduction}

\section{Zero-knowledge Proof and Related Works}

A \textit{proof} is a piece of information that attests to the validity of a statement. First, we would like to narrow down the meaning of \textit{proof} being used in \textit{zero-knowledge proof} and \textit{proof of knowledge}, and provide some insights about these protocols.

We start with Zero-knowledge Proof. Here, we are concerned about \textit{proving membership of NP languages}, for example, prove that a (publicly known) graph G is a Hamiltonian graph.
Step by step, \textit{zero-knowledge} property of a proof can be conceptualized as: \cite{goldreich2007foundations}
\begin{enumerate}
    \item A person who reads the proof (the \textit{verifier}) \textit{gains no knowledge} after reading it.
    \item Gaining knowledge is then defined as: the \textit{computational capability} of the verifier does not increase after reading the proof.
    \item If after reading a proof, the verifier could compute what he could not have \textit{efficiently} computed earlier, then we say the \textit{computational capability} of the verifier has increased, and therefore the proof is \textit{not} zero-knowledge.
\end{enumerate}

Back to our earlier example, if a prover proves that G is a Hamiltonian graph by revealing its Hamiltonian cycle, then the verifier will know the Hamiltonian cycle of G, which cannot be computed efficiently, therefore this proof is \textit{not} Zero-knowledge.

\textbf{Remark:} When we use \textit{efficient}, we mean \textit{polynomial-time}.

Before going to proof of knowledge, we recall an informal definition of NP languages and NP relations. An NP language $L$ is defined as
$$L = \{x: \exists w, |w| \in \mathrm{poly}(|x|), \mathcal{A}(x, w) = 0\}$$
for some deterministic polynomial-time algorithm $\mathcal{A}$. The corresponding relation of $L$ is
$$\mathcal{R}_L = \{ \langle x, w \rangle : \mathcal{A}(x, w) = 0\}$$
$x$ is called \textit{instance} and $w$ is a \textit{witness} for $x$.

A \textit{proof of knowledge} attests that the prover \textit{knows} a witness $w$ for some instance $x$. Proof of knowledge becomes useful when L is the universal set, meaning that every instance is in $L$ and has a witness. Consider, for example, the discrete logarithm problem $\mathcal{R} = \{ \langle x, w \rangle : g^w = x\}$ for a multiplicative group $\G$ and a generator $g$ of $\G$; here, every $x \in \G$ has a discrete logarithm base $g$, therefore, it is not enough to use zero-knowledge proof, we also need a proof of knowledge.

In 1985, Goldwasser, Micali and Rackoff published a paper that marked the beginning of zero-knowledge proof researches \cite{goldwasser1985knowledge}. In this paper, the authors proposed a \textit{zero-knowledge interactive proof} system for languages of Quadratic Residue and Quadratic Nonresidue, i.e. proving that $y$ has (or does not have) a square root modulo $x$. They did this by using an interactive process between the prover and the verifier, where the verifier sends random challenges to the prover.

In 1986, Oded Goldreich et al. proved that all NP languages have zero-knowledge proof system, under the assumption that one-way permutation exists, by constructing a zero-knowledge proof for Graph 3-Coloring problem \cite{goldreich1986prove}. This opens a possibility for a \textit{general purpose} proof system to prove all NP relations in zero-knowledge. However, this result is far from practical, because it is not clear nor intuitive how to transform another NP statement to a graph coloring problem. Recent developments use \textit{arithmetic circuit} to express the statement being proved, which makes it easier to write the statement.

SNARKs are general-purpose proof systems that pursuit the following efficiency characteristics:
\begin{itemize}
    \item \textit{Succinctness}: Proof size is sub-linear to the size of the instance.
    \item \textit{Non-interactivity}: The protocol is 1-round, their is no interaction between the prover and verifiers.
    \item \textit{Verifier performance}: Time-complexity of the verifier is \textit{sub-linear} to the size of its input.
    \item \textit{Prover performance}: Time-complexity of the prover is \textit{linear} or \textit{quasi-linear} to the size of its input.
\end{itemize}

Several SNARKs constructions have been proposed under different \textit{trust models} in the \textit{key generation} step:
\begin{itemize}
    \item \textit{Trusted setup}: The party who generates the key has \textit{toxic wastes} that can compromise the security of the system. In practice, the trusted setup is performed using a \textit{multi-party computation} protocol, such that trust in at least one party is sufficient.
    \item \textit{Transparent setup}: We do not need to trust the party who generate the key.
\end{itemize}

A class of efficient proof systems is \textit{pairing-based} SNARKs, which requires a trusted setup. One such system is Groth16 \cite{groth2016size}, which requires trusted-setup \textit{for each circuit}, has constant proof size of 3 group elements, and proof verification consists of computing 3 pairings. Another class of pairing-based SNARKs are \textit{universal} SNARKs, starting with Sonic \cite{maller2019sonic} and improved in PLONK \cite{gabizon2019plonk} and Marlin \cite{chiesa2020marlin}. Universal SNARKs require a one-time trusted setup to prove all statements up to a certain bound.

Transparent proof systems usually comes with compromises on verifier performance and proof size, two examples are zk-STARKs \cite{ben2019scalable} and Bulletproof \cite{bunz2018bulletproofs}. zk-STARKs has proof size $O(\log^2 T)$, for circuit with 1-million gates, STARK produces a proof with size 600Kb, whereas Groth16 produces proofs with constant-size 200 bytes \cite{bunz2020transparent}. On the other hand, Bulletproof is transparent and has constant-size proofs, however Bulletproof has \textit{linear} verifier time-complexity.

The main study subject of this thesis, Supersonic, is a transparent SNARKs with $O(\log d)$ proof size, $O(\log d)$ verifier time-complexity, and $O(d \log d)$ prover time-complexity.



\section{Organization}

We briefly walk through the contents of each chapter: 

\begin{itemize}
    \item Chapter \ref{chap-background}: \textbf{Background}.
    \begin{itemize}
        \item Basic Abstract Algebra and Computational Complexity theory.
        \item Cryptographic primitives: falsifiable assumptions and commitment schemes.
        \item Defining zero-knowledge proof and proof of knowledge.
    \end{itemize}
    \item Chapter \ref{chap-sonic}: \textbf{Sonic Polynomial Interactive Oracle Proof}.
    \begin{itemize}
        \item Sonic constraints system.
        \item Reconstructing Sonic.
        \item Modifications to Sonic.
    \end{itemize}
    \item Chapter \ref{chap-dark}: \textbf{DARK Polynomial Commitment}.
    \begin{itemize}
        \item Building blocks: homomorphic integer commitments and encoding polynomials as integers.
        \item Commitment scheme for polynomials.
        \item Polynomial evaluation protocol.
        \item Various optimizations to the protocol.
    \end{itemize}
    \item Chapter \ref{chap-circuit}: \textbf{Arithmetic circuits}.
    \begin{itemize}
        \item Definition of circuits and how to compose circuits.
        \item Transforming Boolean circuits to circuits on Finite Field.
        \item How to use arithmetic circuits in Sonic.
        \item Description of SHA-256 and exponentiation-by-squaring circuit.
    \end{itemize}
    \item Chapter \ref{chap-implement}: \textbf{Implementation and Experimentation}.
    \begin{itemize}
        \item Various techniques used in Supersonic implementation.
        \item Source code details.
        \item Experimentation results.
    \end{itemize}
    \item Chapter \ref{chap-conclusion}: \textbf{Conclusion}.
    \begin{itemize}
        \item Summary, open problems, and future works.
    \end{itemize}
\end{itemize}
