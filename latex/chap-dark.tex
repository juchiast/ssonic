\chapter{DARK Polynomial Commitment}
\label{chap-dark}

In this chapter we study DARK Polynomial Commitment scheme introduced by B\"unz et al \cite{bunz2020transparent}. The key contribution of this scheme is that it can be instantiated by a \textit{transparent} Group of Unknown Order, thus making the SNARKs transparent.

We discuss two main constructions: homomorphic commitment and polynomial evaluation protocol with zero-knowledge and witness-extended emulation. The commitment relies on assumptions believed to be hard in generic Group of Unknown Order, namely the Adaptive Root Assumption and the Strong RSA Assumption.

The construction in B\"unz et al. \cite{bunz2020transparent} can be decomposed to:
\begin{enumerate}
    \item An abstract protocol for $\REval$, assuming the existence of a homomorphic commitment for polynomials.
    \item A Homomorphic Integer Commitment in generic Group of Unknown Order.
    \item A scheme to encode polynomials as integers.
    \item $\mathrm{EvalBounded}$: evaluation protocol for $\REval$.
    \item Adding hiding commitments and zero-knowledge evaluation.
    \item Optimizations to reduce proof's size, and to evaluate $m$ polynomials at $k$ points with small overhead.
\end{enumerate}

\section{Homomorphic Polynomial Commitment}

\subsection{Homomorphic Integer Commitment}
\label{section-integer-commitment}

We start with a simple integer commitment, given a group of unknown order $\G$ and an element $g \in \G$, to commit an integer $x$, the committer compute $\gG^x$.

\textbf{Homomorphic Integer Commitment:} Let $\G \gets \mathrm{GGen}(\lambda)$ and $g \rand \G$. We argue that $\gG^x$ is a binding commitment of $x$ if the Strong RSA Assumption holds for $\mathrm{GGen}$. \\
Suppose an adversary $\mathcal{A}$ can break the binding property of the commitment, then $\mathcal{A}$ can find different $x_0$, $x_1$ such that $\gG^{x_0} = \gG^{x_1}$, therefore $\gG^{x_0 - x_1} = 1$. We proceed to construct an adversary for the Strong RSA Assumption, $\mathcal{A}_{RSA}$ picks an odd prime $l$ that is coprime with $x_0 - x_1$, then compute $e \gets l^{-1} \bmod (x_0 - x_1)$, we have $(\gG^e)^l = \gG^{1 + k(x_0 - x_1)} = g$, which mean $\gG^e$ is the $l$-th root of $g$ \cite{bunz2020transparent}. \\
$\gG^x$ is also homomorphic, we have $\gG^{x+y} = \gG^x \gG^y$ and $\gG^{\alpha x} = (\gG^x)^\alpha$.

\subsection{Polynomial Encoding}
\label{section-polynomial-encoding}

DARK made use of the homomorphic integer commitment $\gG^x$ by introducing an polynomial encoding which maps an integer polynomial $f(X) \in \Z[X]$ to an integer.  We summarize their construction below.

\begin{definition}[Polynomial with Bounded Coefficients]
\label{bounded-integer}
Let $\Z(b) = \{x \in \Z : |x| \le b\}$, let $\Z(b)[X] = \{ f(X) = \sum f_i X^i \in \Z[X] : \mathrm{max}\{f_i\} \le b \}$ be the set of integer polynomials with coefficients in $[-b, b]$.
\end{definition}

\textbf{Integer Encoding of Polynomial.} Let $q$ be an odd integer, we define encoding and decoding for polynomials in $\Z(b)[X]$ where $q/2 > b$. Encoding $\aEnc: \Z(b)[X] \to \Z$ and decoding $\aDec: \Z \to \Z[X]$ are defined in algorithm \ref{algo:encode} and \ref{algo:encode}:

\begin{ealgorithm}[Encode polynomial]
\label{algo:encode}
\quad

\textup{
$\aEnc(f(X) \in \Z(q/2)[X]):$
\begin{itemize}
    \item Return $f(q)$.
\end{itemize}
}
\end{ealgorithm}

\begin{ealgorithm}[Decode polynomial]
\label{algo:decode}
\quad

\textup{\textbf{Output:} $f(X) = \sum_{i=0}^{d} f_i X^i$ where $f_i \in [-q/2, q/2]$ for all $i \in [d]$.\\
$\aDec(y \in \Z)$:
\begin{itemize}
    \item We have $|f(q)| = |y| < q^{d+1} \implies \log_q |y| < d + 1$ therefore $d \gets \lfloor log_q |y| \rfloor$.
    \item Then we recover the partial-sum array $S_k = \sum_{i=0}^k f_i q^i$ for $k \in [d]$, observe that
    \begin{itemize}
        \item $S_k \equiv y \pmod {q^{k+1}}$
        \item $|f_i| < q/2 \implies |S_k| < q^{k+1} / 2$
    \end{itemize}
    Therefore,
    $$
    S_k = \begin{cases}
        y \bmod q^{k+1} & \quad \text{if } y \bmod q^{k+1} < q^{k+1} / 2\\
        y \bmod q^{k+1} - q^{k+1} & \quad \text{otherwise}
    \end{cases}
    $$
    \item We have $f_0 = S_0, \forall i > 0, f_i = (S_i - S_{i-1}) / q^i$
\end{itemize}
}
\end{ealgorithm}

\begin{claim}
Let $q$ be an odd positive integer, $\forall f(X) \in \Z(\frac{q-1}{2})[X], \aDec(\aEnc(f(X))) = f(X)$.
\end{claim}

\begin{claim}
Let $q$ be an odd positive integer, for every $z \in \left(-\frac{q^{d+1}}{2}, \frac{q^{d+1}}{2}\right)$, there is a unique polynomial $f(X) \in \Z(\frac{q-1}{2})[X]$ with degree (at most) $d$ such that $f(q) = z$ \cite{bunz2020transparent}. 
\end{claim}

\textbf{Homomorphic properties of $\aEnc$.} $\aEnc$ is homomorphic, we have:
\begin{itemize}
    \item Addition: $f(q) + g(q) = (f + g)(q)$
    \item Scalar multiplication: $\alpha f(q) = (\alpha \cdot f)(q)$ where $\alpha \in \Z$.
\end{itemize}
However, we also need the polynomial to be \textit{uniquely} decode-able, therefore $q$ must be large enough so that the coefficients do not overflow.

\textbf{Encoding for polynomials in $\Z_p[X]$.} A polynomial in $\Z_p[X]$ is encoded by first mapping it into $\Z(\frac{p-1}{2})[X]$, each coefficient $f_i$ is replaced by $f_i \bmod p$ if $f_i \bmod p < p/2$ and replaced by $f_i - p$ otherwise.

\subsection{Binding Commitment for Polynomials}

Combining section \ref{section-integer-commitment} and \ref{section-polynomial-encoding}, we arrive at a Binding Commitment for polynomials in $\Z(q/2)[X]$ and $\Z_p[X]$.

\begin{definition}[Binding commitment for integer polynomials]
\label{dark-commitz}
Let $q$ be an odd positive integer and $p$ is a prime number.
The tuple $\Gamma = (\aSetup, \aCommitZ, \aOpenZ$ is a commitment scheme for polynomials in $\Z(q/2)[X]$.
\begin{itemize}
    \item $\aSetup(1^\lambda) \to \pp$: Sample a group $\G \rand GGen(1^\lambda)$ and a random element $\gG \rand \G$, and return $\pp \gets (\G, \gG, p, q)$.
    \item $\aCommitZ(\pp; f(X) \in \Z(q/2)[X]) \to \gC$: Return $\gC \gets \gG^{f(q)}$.
    \item $\aOpenZ(\pp, f(X), \gC)$: Check that $f(X) \in \Z(q/2)[X]$ and $\gC = \gG^{f(q)}$.
\end{itemize}
\end{definition}

\begin{definition}[Binding commitment for polynomials in $\Z_p$]
\label{dark-commitz}
Let $q$ be an odd positive integer and $p$ is a prime number.
The tuple $\Gamma = (\aSetup, \aCommit, \aOpen)$ is a commitment scheme for polynomials in $\Z_p[X]$.
\begin{itemize}
    \item $\aSetup(1^\lambda) \to \pp$: Sample a group $\G \rand GGen(1^\lambda)$ and a random element $\gG \rand \G$, and return $\pp \gets (\G, \gG, p, q)$.
    \item $\aCommit(\pp; f(X) \in \Z_p[X]) \to (\gC; \hat{f}(X))$: Map $f(X)$ to $\hat{f}(X) \in \Z(\frac{p-1}{2})[X]$, return $\gC \gets \gG^{\hat{f}(q)}$ and $\hat{f}(X)$.
    \item $\aOpen(\pp, f(X), (\gC, \hat{f}(X)))$: Check that $\hat{f}(X) \in \Z(q/2)[X]$ and $\gC = \gG^{\hat{f}(q)}$ and $f_i \equiv \hat{f}_i \pmod p$.
\end{itemize}
\end{definition}

\begin{theorem}
\label{claim-binding}
$\Gamma = (\aSetup, \aCommitZ, \aOpenZ)$ is \textup{binding} if the Adaptive Root Assumption and the Strong RSA Assumption hold for $GGen$. \cite{bunz2020transparent}.
\end{theorem}

\textbf{Proof of theorem \ref{claim-binding}.} Suppose an adversary $\mathcal{A}$ breaks the binding property of $\Gamma$, which means $\mathcal{A}$ can produce $f_0(X) \ne f_1(X)$ such that $\gG^{f_0(q)} = \gG^{f_1(q)} = \mathrm{C}$, then:
\begin{itemize}
    \item $\gG^{f_0(q) - f_1(q)} = 1$ which breaks the Low Order assumption and therefore breaks the Adaptive Root assumption.
    \item We construct adversary $\mA_{RSA}$ for the Strong RSA assumption. $\mathcal{A}_{RSA}$ pick an odd prime $l$ that is coprime to $f_0(q)-f_1(q)$, then compute $e \gets l^{-1} \bmod (f_0(q)-f_1(q))$, we have $(\gG^e)^l = \gG^{1 + k(f_0(q)-f_1(q))} = g$, therefore $\gG^e$ is the $l$-th root of $g$ and $\mathcal{A}_{RSA}$ break the Strong RSA assumption.
\end{itemize}

\subsection{Hiding Commitment for Polynomials}

\textit{Hiding} property is added to $\aCommitZ$ by adding a large blinding coefficients. Let $d \gets \deg(f(X))$, then the commitment is $\gG^{f(q) + r q^{d+1}}$ for a large $r$.

\begin{definition}[Hiding Polynomial Commitment]
Let $B \ge |\G|$ be a publicly known bound of the order of $\G$.
\begin{itemize}
    \item $\aSetup(1^\lambda) \to \mathrm{pp}$: Sample a group $\G \rand GGen(1^\lambda)$ and a random element $\gG \rand \G$, let $B > |\G|$ be a publicly known bound of the order of $\G$, return $\mathrm{pp} \gets (\G, \gG, p, q, B)$.
    \item $\aHidingCommit(\mathrm{pp}; f(X) \in \Z(q/2)[X]) \to (\gC; r, d)$: Let $d \gets \mathrm{deg}(f(X))$ and $r \rand [0, B \cdot 2^\lambda)$, return commitment $\gC \gets \gG^{f(q) + r \cdot q^{d+1}}$ with opening hint $r$ and $d$.
    \item $\aHidingOpen(\mathrm{pp}, f(X), \mathrm{C}, r, d)$: Check that $f(X) \in \Z(q/2)[X]$ and $\gC = \gG^{f(q) + r \cdot q^{d+1}}$ and $\mathrm{deg}(f(X)) \le d$.
\end{itemize}
\end{definition}

\textbf{Remark:} Hiding commitment for polynomial in $\Z_p[X]$ is defined similarly.

\begin{theorem}
The tuple $\Gamma = (\aSetup, \aHidingCommit, \aHidingOpen)$ is \textup{hiding} if the \textup{Subgroup Indistinguishability assumption} holds for $\aSetup$ \cite{bunz2020transparent}.
\end{theorem}

\begin{definition}[Subgroup Indistinguishability assumption \cite{bunz2020transparent}]
The \textup{Subgroup Indistinguishability assumption} hold for $\aSetup$ if no efficient adversary can distinguish between a random element of $\langle \mathsf{h} \rangle$ and $\langle \gG \rangle$ for any non-trivial $\mathsf{h} \in \langle \gG \rangle$.
\end{definition}

\subsection{Analysis and Optimizations}

\textbf{Homomorphic properties.} We have three homomorphisms for the commitment scheme, denote $\C{f(X)}$ as the commitment of $f(X)$, $\C{\cdot}: \Z(q/2)[X] \to \G$, we have:
\begin{itemize}
    \item Addition: $\C{f(X) + h(X)} = \gG^{f(q) + h(q)} = \gG^{f(q)} \gG^{h(q)} = \C{f(X)}\C{h(X)}$.
    \item Scalar multiplication: $\C{\alpha f(X)} = \gG^{\alpha f(q)} = (\gG^{f(q)})^{\alpha} = \C{f(X)}^{\alpha}$.
    \item Monomial multiplication: $\C{X^k f(X)} = \gG^{f(q) \cdot q^k} = \C{f(X)}^{q^k}$
\end{itemize}
These homomorphisms hold as long as the original and resulted polynomials are in $\Z(q/2)[X]$.

\textbf{Optimizations.} Computing $\gG^{f(q)}$ takes $O(d\log q)$ group multiplications, where $d$ is the degree of the polynomial. We could bring this down with pre-computation, observing that $\gG^{f(q)} = \prod_{i=0}^d (\gG^{q^i})^{f_i}$, we can pre-compute $\gG^{q^i}$ for all $i \in [d]$, the cost is now $O(d\log \mathrm{max}\{f_i\})$. We will see in the construction of evaluation protocol that $q$ is much larger than the coefficients of $f(X)$, $q \in O(p^{\log d})$, therefore this brings the cost down by a $\log d$ factor.

Furthermore, we can compute $\prod_{i=0}^d (\gG^{q^i})^{f_i}$ in parallel, and multi-exponentiation techniques \cite{yen2012multi, borges2017parallel} can be employed.

\section{Polynomials Evaluation Protocol}

This section describes the \textit{zero-knowledge argument of knowledge} for the relation $\REval$ introduced by B\"unz et al. \cite{bunz2020transparent}. Informally, given $\gG \in \G, \mathrm{C} \in \G$ $y, z \in \Z_p$, $d \in \N$, the argument proves that the Prover knows $f(X) \in \Z((p-1)/2)[X]$ such that:
\begin{itemize}
    \item $\gC$ is the commitment of $f(X)$: $\gC = \gG^{f(q)}$
    \item $\deg(f(X)) \le d$
    \item $f(z) \equiv y \pmod p$
\end{itemize}

\subsection{Protocol Description}

DARK does this by splitting the polynomial in half and taking a random linear combination to reduce the statement about $f(X)$ with degree $d$ to $f'(X)$ with degree $d/2$. When $d = 0$, $f(X)$ is a constant and $\mP$ sends it directly to $\mV$.

Protocol \ref{dark-eval} is an interactive argument for $\REval$, which uses recursive protocol \ref{dark-evalbounded} as a sub-protocol.

\begin{protocol}[Eval]
\label{dark-eval} \quad

\textup{
$\aEval(\pp, \mathrm{C} \in \G, z \in \Z, y \in \Z, d \in \N; f(X) \in \Z_p[X]):$
\begin{enumerate}
    \item $\mP:$ lift $f(X)$ to $\hat{f}(X) \in \Z(\frac{p-1}{2})[X]$.
    \item $\mP$ and $\mV$: let $\pp \gets (\pp, \varsigma)$, where $\varsigma \gets p^{\log_2(d+1)}$ if $\G$ is RSA group and $\varsigma \gets p^{2\log_2(d+1)}$ if $\G$ is class group.
    \item $\mP$ and $\mV$ run $\aEvalBounded(\pp, \gC, z, y, d, (p-1)/2; \hat{f}[X])$.
\end{enumerate}
}
\end{protocol}

\begin{protocol}[EvalBounded]
\label{dark-evalbounded} \quad

\textup{
$\aEval(\pp, \mathrm{C} \in \G, z \in \Z, y \in \Z, d \in \N, b \in \N; f(X) \in \Z(b)[X]):$
\begin{itemize}
    \item If $d = 0$:
    \begin{itemize}
        \item $\mP$: send $f$, where $f(X) = f$.
        \item $\mV$: check that $|f| < b$ and $f \equiv y \pmod p$ and $\gG^{f} = \gC$ and $b \cdot \varsigma < q$. Output 1 if all checks pass, otherwise output 0. 
    \end{itemize}
    \item If $d$ is even:
    \begin{itemize}
        \item $\mP$: let $f'(X) \gets X f(X)$.
        \item $\mV$: let $\gC' \gets \gC^q, y' \gets yz, d' \gets d + 1$.
        \item $\mP$ and $\mV$ run $\aEvalBounded(\pp, \gC', z, y', d', b; f'(X))$.
    \end{itemize}
    \item If $d$ is odd:
    \begin{itemize}
        \item $\mP$ and $\mV$: let $d' \gets \lfloor d/2 \rfloor$.
        \item $\mP$: compute $f_L(X) \gets \sum_{i=0}^{d'} f_i X^i$ and $f_H(X) \gets \sum_{i=0}^{d'} f_{d' + i + 1} X^i$, $\gC_L \gets \gG^{f_L(q)}$, $\gC_H \gets \gG^{f_H(q)}$, $y_L \gets f_L(z) \bmod p$, $y_H \gets f_H(z) \bmod p$. $\mP$ sends $\gC_L, \gC_H, y_L, y_H$.
        \item $\mV$: check that $y \equiv y_L + z^{d'+1} y_H \pmod p$.
        \item $\mP$ and $\mV$ run $\aPoE(\gC_H, \gC / \gC_L, q^{d'+1})$.
        \item $\mV$: sample $\alpha \rand[-\frac{p-1}{2}, \frac{p-1}{2}]$ and send it to $\mP$.
        \item $\mP$ and $\mV$: let $\gC' \gets \gC_L^\alpha \gC_H$, $y' \gets (\alpha y_L + y_R) \bmod p$, $b' \gets b 2^{\lambda + 1}$.
        \item $\mP$ and $\mV$ run $\aEvalBounded(\pp, \gC', z, y', d', b; f'(X))$.
    \end{itemize}
\end{itemize}
}
\end{protocol}

\begin{protocol}[Proof of Exponentiation]
\label{aPoE}
\quad

\textup{
$\aPoE(u \in \G, w \in \G, e \in \N)$:
\begin{itemize}
    \item $\mV$: sample random $O(\lambda)$-bit prime $l$ and send it to $\mP$.
    \item $\mP$ and $\mV$: perform Euclidean division $e = k l + r$.
    \item $\mP$: compute $x \gets u^k$ and send it to $\mV$.
    \item $\mV$: check that $x^l u^r = w$.
\end{itemize}
}
\end{protocol}

\begin{theorem}[PoE soundness \cite{bunz2020transparent}]
$\aPoE$ is an interactive argument for the relation $\RPoE = \{\langle (u, w, e), \varnothing \rangle : u, w \in \G, e \in N, u^e = w \}$ with negligible soundness error if the Adaptive Root Assumption holds for $GGen$.
\end{theorem}

\textbf{$\aEval$ soundness.} There are two theorems in \cite{bunz2020transparent} to show that $\aEval$ has Witness-Extended Emulation, one for RSA groups and one for class groups. Witness-Extended Emulation implies soundness, therefore we do not need a soundness proof for the protocol. Theorem \ref{wee-rsa} is for RSA groups, the theorem uses Strong RSA Assumption, this assumption does not hold for class groups. Therefore theorem \ref{wee-class} replaces it with 2-Strong RSA Assumption, this increases the bound on $q$.

\begin{theorem}
\label{wee-rsa}
The tuple $\Gamma = (\aSetup, \aCommit, \aOpen, \aEval)$ has Witness-Extended Emulation for the relation $\REval$ if $q > p^{2\lceil \log_2(d+1) \rceil + 1}$ and the Adaptive Root Assumption and the Strong RSA Assumption hold for $GGen$, where $d$ is the maximum degree of polynomials that $\Gamma$ can process.
\end{theorem}

\begin{theorem}
\label{wee-class}
The tuple $\Gamma = (\aSetup, \aCommit, \aOpen, \aEval)$ has Witness-Extended Emulation for the relation $\REval$ if $q > p^{3\lceil \log_2(d+1) \rceil + 1}$ and the Adaptive Root Assumption and the 2-Strong RSA Assumption hold for $GGen$, where $d$ is the maximum degree of polynomials that $\Gamma$ can process.
\end{theorem}

\subsection{Optimizations}
\label{sec-dark-optimization}

\textbf{Use PoE to transfer the commitments.} Instead of sending $\gC_L$ and $\gC_H$ then using PoE to prove that $\gC / \gC_L = \gC_H^{q^{d'+1}}$, we can use PoE to transfer the commitments. The modification to $\aEval$ is as follow:
\begin{itemize}
    \item $\mP$: send $\gC_H$ to $\mV$.
    \item $\mV$: sample random $O(\lambda)$-bit prime $l$ and send it to $\mP$.
    \item $\mP$ and $\mV$: perform Euclidean division $q^{d'+1} = k l + r$.
    \item $\mP$: send $x \gets \gC_H^k$ to $\mV$.
    \item $\mV$: let $\gC_L \gets \gC / (x^l \gC_H^r)$.
\end{itemize}
This optimization reduce the proof size by one element per round, for RSA group of 120-bit security, one element is worth 3048 bits.

Similarly, $\mV$ can infer $y_L$: $y_L \gets y - z^{d' + 1} y_H$.

\textbf{Compute PoE in parallel.} PoE contains a large exponentiation $\gC_H^k$ where $k = \lfloor q^{d' + 1} / l \rfloor$ has $O(d' \log_2 q)$ bits. Observing that $C_H = \gG^{f_H(q)}$, we rewrite $k$ as a polynomial $h(X)$ such that $h(q) = k$, then we have $\gC_H^k = \gG^{f_H(q) \cdot h(q)} = \gG^{(f_H \cdot h)(q)}$, then pre-computation of $\gG^{q^i}$ can be used to compute the exponentiation in parallel.\\
Let $h(X) = \sum_{i=0} h_i X^i$, we describe how to find $h(X)$ such that $h(q) = \lfloor q^d / l \rfloor$. Assuming that $0 < x < l$, we have:
$$\frac{xq^d}{l} = \lfloor xq / l \rfloor q^{d-1} + \frac{(xq \bmod l)q^{d-1}}{l}$$
Using the above formula, start with $x = 1$, we can extract the coefficients of $h(X)$, starting from the most-significant coefficient.

\textbf{Evaluating at multiple points.} The protocol can be extended to evaluating at $k$ points, $\Vec{z}, \Vec{y} \in \F^k$. At the final step, the verifier checks that $f \equiv y_i \pmod p$ for all $i \in [k]$.

\textbf{Combining statements.} It is also possible to combine two statements $f(z) = y_0$ and $g(z) = y_1$ into a single statement $h(z) = y_2$ by taking linear combination of $f(X)$ and $g(Z)$. In general, we can evaluate $m$ polynomials by combining it into a single polynomials \cite{bunz2020transparent}. To reserve the degree bound condition, each polynomial is shifted to the maximum degree in $k$ polynomials. We present the protocol for evaluating $m$ polynomials in protocol \ref{joined-eval}.
\begin{protocol}[JoinedEval]
\label{joined-eval}
\quad

\textup{$\aJoinedEval(\pp, z \in \Z, \{\gC_i, y_i \in \Z, d_i \in \N\}_{i=1}^{m}; \{\hat{f}_i(X) \in \Z_p[X]\}_{i=1}^{m}):$
\begin{itemize}
    \item Let $d \gets \max \{d_i\}$
    \item For each $i \in [m]$;
    \begin{itemize}
        \item $\mP$: Lift $\hat{f}_i(X) \in \Z_p[X]$ to $f_i(X) \in \Z(\frac{p-1}{2})[X]$, let $f_i(X) \gets X^{d - d_i} f_i(X)$.
        \item $\mV$: Let $y_i \gets z^{d - d_i} y_i$, $d_i = d$.
        \item $\mP$: Compute $u \gets \gC^{q^{d - d_i}}$ and send $u$ to $\mV$.
        \item $\mP$ and $\mV$ run $\aPoE(\gC_i, u, q^{d - d_i})$ to check the correctness of $u$, if the check pass then let $\gC_i \gets u$.
    \end{itemize}
    \item $\mV$: Sample $\Vec{c} \rand [-\frac{p-1}{2}, \frac{p-1}{2}]^{m-1}$
    \item $\mP$: Let $f(X) = f_m(X) + \sum_{i=1}^{m-1} c_i f_i(X)$.
    \item $\mV$: Let $y \gets y_m + \sum_{i=1}^{m-1} c_i y_i$, $\gC \gets C_m \prod_{i=1}^{m-1}C_i^{c_i}$, $b \gets m \frac{p-1}{2} \cdot 2^\lambda$.
    \item $\mP$ and $\mV$ run $\aEvalBounded(\pp, \gC, z, y, d, b; f(X))$.
\end{itemize}
}
\end{protocol}

\subsection{Zero-knowledge Evaluation}

To achieve zero-knowledge, \cite{bunz2020transparent} uses \textit{hiding commitment} in combination with a \textit{random blinding polynomial}.

\begin{protocol}[ZKEval]
\label{zk-eval}
\quad

\textup{
$\aZKEval(\pp, \gC \in \G, z \in \Z, y \in \Z, d \in \N, b \in \Z; f(X) \in \Z(b)[X], r \in \Z)):$
\begin{itemize}
    \item $\mP$: sample a random degree $d$ polynomial $k(X) \rand \Z(b \cdot \frac{p-1}{2} \cdot 2^\lambda)[X]$, compute hiding commitment $(\gC_k, r_k) \gets \aHidingCommit(\pp; k(X))$, let $y_k \gets k(z) \bmod p$, send $\gC_k$ and $y_k$ to $\mV$.
    \item $\mV$: sample random $c \rand [-\frac{p-1}{2}, \frac{p-1}{2}]$.
    \item $\mP$: compute $s(X) \gets k(X) + c \cdot f(X)$ and $r_s \gets r_k + c \cdot r$, send $r_s$ to $\mV$.
    \item $\mV$: compute $\gC_s \gets \gC_k \gC^c \cdot \gG^{-r_s \cdot q^{d+1}}$, $y_s \gets y_k + c \cdot y$, $b_s \gets b \cdot \frac{p-1}{2} \cdot (2^\lambda + 1)$.
    \item $\mP$ and $\mV$ run $\aEvalBounded(\pp, \gC_s, z, y_s, d, b_s; s(X))$.
\end{itemize}
}
\end{protocol}

Basically, $\mP$ samples a random $k(X)$ and releases its hiding commitment $\gC_k$, $\mV$ sample a challenge $c$, $\mP$ mix $k(X)$ and $f(X)$ into $s(X)$ and release its blinding coefficient $r_s$, $\mV$ then derives a non-hiding commitment of $s(X)$ and use it in $\aEvalBounded$. Zero-knowledge comes from the fact that $s(X)$ is statistically close to a random polynomial, the challenge $c$ keeps it from being totally random, therefore ensure the soundness of the protocol.

\textbf{Evaluate multiple polynomials in zero-knowledge.} We can combine the polynomials as in section \ref{sec-dark-optimization}, the new blinding coefficient is $r \gets r_m + \sum_{i=1}^{m-1} c_i r_i$.

\begin{theorem}[Security of $\aZKEval $\cite{bunz2020transparent}]
Protocol $\aZKEval$ has \textup{witness-extended emulation} and \textup{honest-verfier $\delta$-statistical zero-knowledge} for $q > p^{2 \log_2 d + 4}$ and $\delta < d \cdot 2^{-\lambda}$, assuming the Adaptive Root assumption and the strong RSA assumption holds.
\end{theorem}

\textbf{Proof sketch.} We show that $\aZKEval$ has honest-verifier statistical zero-knowledge.\\
Let $\aView(\mP(x, w), \mV(x))$, be the distribution of the view of $\mV$ when interacting with $\mP$, on common input $x$ and public parameter $\pp$. We construct a PPT simulator $\mS$ such that $\mS(x)$ is statistically close to $\aView(\mP(x, w), \mV(x))$.\\
$\aView(\dots)$ consists of $\gC_k, y_k, c, \gC_s, y_s$ and $\mV$'s view in $\aEvalBounded$. Given the instance $x = (\gC, z, y)$, $\mS$ follow the steps:
\begin{enumerate}
    \item Sample uniformly random $s(X) \rand \Z(b_s)[X]$ and random blinding coefficient $r_s \gets [0, B \cdot 2^\lambda \cdot \frac{p-1}{2}]$.
    \item Sample random challenge $c \rand [-\frac{p-1}{2}, \frac{p-1}{2}]$.
    \item Compute commitment of $s(X)$, $\gC_s \gets \gG^{s(q)}$.
    \item Compute hiding commitment of $k(X)$, $\gC_k \gets \gC^{-c} \cdot \gG^{r_s \cdot q^{d+1}}$.
    \item Compute $y_k \gets s(z) - c \cdot y$.
    \item $\mS$ run $\aEvalBounded$ with $s(X)$ as the witness.
\end{enumerate}
$\mS$ returns a valid transcript and is statistically close to the distribution of the real transcript \cite{bunz2020transparent}.

\section{Proving Witness-extended Emulation}

Beside the Adaptive Root assumption and the Strong RSA assumption, DARK requires a bound on the parameters $p$ and $q$, e.g. $q > p^{2 \cdot \log_2 d + 2}$ for $\aEval$ and $q > p^{2 \cdot \log_2 d + 4}$ for $\aZKEval$ on RSA group. We will briefly show how to get these bounds, for the full proof, prefer to \cite{bunz2020transparent}.

In \cite{bunz2020transparent}, witness-extended emulation was proven using the General Forking lemma, by constructing an extractor $\mX$ that, given a complete binary tree of \textit{accepting} transcripts, output a witness $f(X)$ for the interaction. 

\textbf{Binary tree of transcripts.} At the root of the tree is the instance $(\gC, z, y)$, whenever the verifier sample the challenge $\alpha$, the node splits into two child nodes, each node represents a different value for $\alpha$. At the leaf of the tree is the constant $f$ that the prover sent. For a polynomial of degree $d$, the number of challenges is $n = \lceil \log_2 (d+1) \rceil$, we have a tree of $2^n \in O(d)$ possible transcripts.

We will be analyzing each step of the recursive function $\aEvalBounded$. In round $i$, the instance is:
\begin{itemize}
    \item $\gC^{(i)} \in \G$: commitment of the witness that we need to extract.
    \item $y^{(i)} \in \Z$.
    \item $d^{(i)}$: degree of the witness.
    \item $z \in \Z$: evaluation point, unchanging throughout the protocol.
\end{itemize}

Denote by $f^{(i)}(X)$ the witness that we need to extract at each round $i$, $f^{(i)}(X)$ has to satisfy:
\begin{itemize}
    \item $\gG^{f^{(i)}(q)} = \gC^{(i)}$
    \item $f^{(i)}(z) = y^{(i)}$
    \item $\deg(f^{(i)}(X)) \le d^{(i)}$
\end{itemize}

Starting from the leaf of the tree, where the prover sent a constant $f \in \Z$, we have that $f^{(n)}(X) = f$, where $n$ is the height of the tree. Because this is an \textit{accepting} transcript, $f^{(n)}(X)$ satisfies all of the conditions above.

When $d^{(i)}$ is even, assuming we have the extracted witnesses of the next round $f^{(i+1)}(X)$. We have that $f^{(i+1)}(q)$ must be divisible by $q$, otherwise $\gC^{(i)}$ is the $q$-th root of $\gG^{f^{(i+1)}(q)}$ and we break the Fractional Root assumption.

When $d^{(i)}$ is odd, the tree splits into two child nodes, each node with a different challenge $\alpha_1, \alpha_2 \in [-\frac{p-1}{2}, \frac{p-1}{2}]$. According to lemma \ref{lemma:combining}, we can extract the witness $f^{(i)}(X)$ from two child witnesses $f^{(i+1)}_1(X)$ and $f^{(i+1)}_2(X)$.

\begin{lemma}[Combining for integer witnesses \cite{bunz2020transparent}]
\label{lemma:combining}
Let $(\G, \gG, q, p)$ be the parameters of DARK commitment scheme. Given $(z, \gC_L, \gC_H, y_L, y_H)$ and $(\alpha, f, y, \alpha', f', y')$ such that $\gG^f = \gC_L^\alpha \cdot \gC_H$ and $\gG^{f'} = \gC_L^{\alpha'} \cdot \gC_H$. Further, let $f$ and $f'$ decode to $f(X)$ and $f'(X)$ with coefficients bounded by $b$, i.e. $f(X) \gets \aDec(f), f'(X) \gets \aDec(f')$, and $f(X), f'(X) \in \Z(b)[X]$. Let $f(z) \equiv y \pmod p$ and $f'(z) \equiv y' \pmod p$. Then, there exists an PPT extractor $\mX$ that can compute either:
\begin{enumerate}
    \item $f_L(X), f_H(X) \in \Z(b \cdot (p-1))[X]$ such that
    \begin{itemize}
        \item $f_L(z) \equiv y_L \pmod p$ and $f_H(z) \equiv y_H \pmod p$.
        \item $\gC_L = \gG^{f_L(q)}$ and $\gC_H = \gG^{f_H(q)}$.
    \end{itemize}
    \item A break of the Low Order assumption.
    \item A break of the Fractional Root assumption.
\end{enumerate}
\end{lemma}

\textbf{Infer the bound of $q$.} $q$ has to be large enough so that decoding works correctly during witnesses extraction, concretely, $q/2$ must be larger than all coefficients of extracted polynomials. In $\aEvalBounded$, whenever $\mV$ sample a challenge $\alpha \in [-\frac{p-1}{2}, \frac{p-1}{2}]$, the coefficients bound is increased by a factor of $p$, application of lemma \ref{lemma:combining} increased it by a factor of $p$, therefore we have $q > p \cdot p^{2 \lceil \log_2 (d+1) \rceil}$ in theorem \ref{wee-rsa}. 

\section{Generating and Verifying Parameters}

We describe how to sample $(p, q, \G, \gG)$ to be used in the the commitment scheme, we call this procedure $\aParamGen$.

\textbf{Input.} Input of $\aParamGen$ are:
\begin{itemize}
    \item Security parameter $\lambda = 120$.
    \item $m$: maximum number of polynomials to be combined using $\aJoinedEval$.
    \item $d$: maximum degree of input polynomials.
\end{itemize}

\textbf{Output.} Output of $\aParamGen$ are:
\begin{itemize}
    \item $p, q, \G, \gG$.
    \item $(\gG^q)^i$ for each $i \in [d]$.
    \item Proof of correct exponentiation of each $(\gG^q)^i$.
\end{itemize}

$\aParamGen$ samples a 3048-bit RSA group $\gG$, a random element $\gG \in \G$. 3048-bit RSA groups are believed to have 120-bit security \cite{bunz2020transparent}.

Then, $\aParamGen$ samples a $(\lambda + \log_2 d)$-bit prime number, and sample an odd number $q$ that has $|p| \cdot (2 \cdot \lceil \log_2 d \rceil + 4 + m)$ bits, where $|p|$ is the bit-size of $p$. This ensure $q > p^{2 \cdot \log_2 d + 4 + m}$, therefore we can evaluate $m$ polynomials in zero-knowledge using there parameters.

Then, $\aParamGen$ compute each $(\gG^q)^i$ and use $\aPoE$ to produce a proof for this computation, this proof ensures the correctness of the values and is fast to verify.