\begin{EnAbstract}
SNARKs is short for \textbf{S}uccinct \textbf{N}on-interactive (zero-knowledge) \textbf{AR}gument of \textbf{K}nowledge. \textit{Zero-knowledge Argument of Knowledge} is a protocol in which the Prover tries to convince the Verifier that (1) he knows a witness $w$ for an instance $x$ of an NP relation $\mathcal{R}$ (2) without revealing anything else about $w$. (1) is the property of \textit{Proof of Knowledge} and (2) is a \textit{Zero-knowledge 
Proof}. In this thesis, we study the notion of \textit{Zero-knowledge Proof} and \textit{Proof of Knowledge}, and how to define and prove the security of such schemes.

We also study Supersonic, a state-of-the-art SNARKs introduced by Bünz et al. \cite{bunz2020transparent}. Supersonic removes a major obstacle in deploying SNARKs: the need for a \textit{trusted} setup. In recent years, SNARKs have been used in blockchain networks to add privacy to the chain, for instance, ZCash uses SNARKs to hide the amount and addresses of transactions. The \textit{decentralized} nature of blockchain networks discourages the usage of any protocols with \textit{trusted} setup, where parties involved must successfully destroy the \textit{toxic wastes} generated while setting up the protocol, as anyone who possesses the toxic wastes can break the security of the system. In fact, ZCash had to do an elaborate "ceremony" where the participants had to buy a computer from a random store, remove all network adapters and hard disks before booting the computer \footnote{\url{https://z.cash/technology/paramgen/}}.

Supersonic is a combination of DARK Polynomial Commitment scheme \cite{bunz2020transparent} and the underlying abstract protocol of Sonic \cite{maller2019sonic}. This thesis provides (1) a description of DARK with all suggested optimizations, (2) a reconstruction of Sonic as a Polynomial IOP, (3) some changes to make Sonic compatible with DARK, and (4) an implementation of Supersonic. Furthermore, we show (5) how to compose arithmetic circuits and how to use them in Supersonic. Finally, we provide (6) experimentation results of Supersonic on \textit{non-padding SHA-256} circuit and \textit{exponentiation by squaring} circuit.

\end{EnAbstract}